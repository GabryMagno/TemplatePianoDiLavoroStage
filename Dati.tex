%----------------------------------------------------------------------------------------
%   USEFUL COMMANDS
%----------------------------------------------------------------------------------------

\newcommand{\dipartimento}{Dipartimento di Matematica ``Tullio Levi-Civita''}

%----------------------------------------------------------------------------------------
% 	USER DATA
%----------------------------------------------------------------------------------------

% Data di approvazione del piano da parte del tutor interno; nel formato GG Mese AAAA
% compilare inserendo al posto di GG 2 cifre per il giorno, e al posto di 
% AAAA 4 cifre per l'anno
\newcommand{\dataApprovazione}{Data}

% Dati dello Studente
\newcommand{\nomeStudente}{Gabriele Isacco}
\newcommand{\cognomeStudente}{Magnelli}
\newcommand{\matricolaStudente}{2075542}
\newcommand{\emailStudente}{gabrieleisacco.magnelli@studenti.unipd.it}
\newcommand{\telStudente}{+ 39 347 317 0439}

% Dati del Tutor Aziendale
\newcommand{\nomeTutorAziendale}{Roberto}
\newcommand{\cognomeTutorAziendale}{Martina}
\newcommand{\emailTutorAziendale}{ingmcrm@gmail.com}
\newcommand{\telTutorAziendale}{+ 39 389 788 7744}
\newcommand{\ruoloTutorAziendale}{}

% Dati dell'Azienda
\newcommand{\ragioneSocAzienda}{CWBI}
\newcommand{\indirizzoAzienda}{Via Venezia 92/B, Padova (PD)}
\newcommand{\sitoAzienda}{https://www.cwbi.eu/it/}
\newcommand{\emailAzienda}{info@cwbi.it}
\newcommand{\partitaIVAAzienda}{P.IVA 12345678999}

% Dati del Tutor Interno (Docente)
\newcommand{\titoloTutorInterno}{Prof.}
\newcommand{\nomeTutorInterno}{Davide}
\newcommand{\cognomeTutorInterno}{Bresolin}

\newcommand{\prospettoSettimanale}{
     % Personalizzare indicando in lista, i vari task settimana per settimana
     % sostituire a XX il totale ore della settimana
    \begin{itemize}
        \item \textbf{Prima Settimana - Fondamenti e preparazione (41 ore)}
        \begin{itemize}
            \item Studio di ML, NLP e LLM applicati al codice;
            \item Installazione ambienti, librerie Python e IDE;
            \item Familiarizzazione con repository e pipeline CI/CD;
            \item Documentazione dell’ambiente e primi esperimenti;
            \item Formazione sulle tecnologie adottate;
        \end{itemize}
        \item \textbf{Seconda Settimana - Analisi del codice e raccolta dati (38 ore)} 
        \begin{itemize}
            \item Raccolta di codice conforme/non conforme;
            \item Annotazione manuale degli errori e non conformità;
            \item Creazione dataset iniziale pronto per ML;
        \end{itemize}
        \item \textbf{Terza Settimana - Preprocessing e rappresentazione del codice (38 ore)} 
        \begin{itemize}
            \item Tokenizzazione e pulizia del dataset;
            \item Studio e implementazione di embedding per il codice;
            \item Automatizzazione preprocessing e validazione dataset;
        \end{itemize}
        \item \textbf{Quarta Settimana - Addestramento iniziale del modello (38 ore)} 
        \begin{itemize}
            \item Configurazione modello LLM e parametri di training;
            \item Addestramento su subset del dataset;
            \item Monitoraggio delle metriche e prima valutazione;
        \end{itemize}
        \item \textbf{Quinta Settimana - Fine-tuning e ottimizzazione (38 ore)} 
        \begin{itemize}
            \item Fine-tuning su dataset specifico aziendale;
            \item Analisi dei casi di errore e miglioramento iterativo;
            \item Tecniche di data augmentation;
            \item Studio di interpretabilità del modello;
        \end{itemize}
        \item \textbf{Sesta Settimana - Integrazione in pipeline CI/CD (38 ore)} 
        \begin{itemize}
            \item Studio e configurazione della pipeline CI/CD;
            \item Integrazione del modello e scripting per test automatici;
            \item Logging, gestione feedback e validazione preliminare;
        \end{itemize}
        \item \textbf{Settima Settimana - Validazione e test delle correzioni (36 ore)} 
        \begin{itemize}
            \item Test su nuovo codice non incluso nel training;
            \item Valutazione metriche di accuratezza e recall;
            \item Iterazioni per miglioramento del comportamento del modello;
        \end{itemize}
        \item \textbf{Ottava Settimana - Documentazione e presentazione finale (32 ore)} 
        \begin{itemize}
            \item Redazione documentazione tecnica completa;
            \item Creazione esempi pratici e guide d’uso;
            \item Preparazione presentazione finale al team;
            \item Presentazione e discussione dei risultati;
        \end{itemize}
    \end{itemize}
}

% Indicare il totale complessivo (deve essere compreso tra le 300 e le 320 ore)
\newcommand{\totaleOre}{}