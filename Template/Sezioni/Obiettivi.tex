%----------------------------------------------------------------------------------------
%	OBJECTIVES
%----------------------------------------------------------------------------------------
\section*{Obiettivi}
\subsection*{Notazione}
Si farà riferimento ai requisiti secondo le seguenti notazioni:
\begin{itemize}
	\item \textit{O} per i requisiti obbligatori, vincolanti in quanto obiettivo primario richiesto dal committente;
	\item \textit{D} per i requisiti desiderabili, non vincolanti o strettamente necessari,
		  ma dal riconoscibile valore aggiunto;
	\item \textit{F} per i requisiti facoltativi, rappresentanti valore aggiunto non strettamente 
		  competitivo.
\end{itemize}

Le sigle precedentemente indicate saranno seguite da una coppia sequenziale di numeri, identificativo del requisito.
\newpage
\subsection*{Obiettivi fissati}
Si prevede lo svolgimento dei seguenti obiettivi:
\begin{center}
    \begin{tabularx}{\textwidth}{|c|X|X|}
    \hline
    \textbf{Requisiti (O,D,F)} & \textbf{Attività} & \textbf{Descrizione} \\
    \hline
    
    % --- Obbligatori ---
    \multicolumn{3}{|c|}{\textbf{Requisiti Obbligatori}} \\
    \hline

    O1 & Analisi del codice & Il modello deve essere in grado di leggere e interpretare codice sorgente prodotto dagli sviluppatori \\
    \hdashline

    O2 & Rilevamento delle non conformità & Il modello deve individuare errori o deviazioni dagli standard e best practice aziendali \\
    \hdashline

    O3 & Segnalazione automatica degli errori & Ogni non conformità deve essere riportata chiaramente agli sviluppatori \\
    \hdashline

    O4 & Addestramento con dataset etichettato & Il progetto deve prevedere la raccolta e l’annotazione di codice conforme/non conforme \\
    \hdashline

    O5 & Integrazione nella pipeline CI/CD & Il modello deve essere utilizzabile in contesti di sviluppo reale senza bloccare i processi \\
    \hdashline

    O6 & Test e validazione & Le correzioni proposte devono essere testate per verificare efficacia e sicurezza \\
    \hdashline

    O7 & Documentazione & Fornire una documentazione completa per l’uso e la manutenzione del sistema \\
    \hline

    % --- Desiderabili ---
    \multicolumn{3}{|c|}{\textbf{Requisiti Desiderabili}} \\
    \hline

    D1 & Correzione automatica degli errori & Il modello dovrebbe proporre e, se possibile, applicare correzioni al codice \\
    \hdashline

    D2 & Espandibilità e aggiornabilità del dataset & Il sistema deve permettere di aggiornare facilmente il modello con nuovi snippet o standard \\
    \hdashline

    D3 & Interpretabilità delle decisioni & Il modello dovrebbe fornire spiegazioni sul perché segnala un errore o propone una correzione \\
    \hline

    % --- Facoltativi ---
    \multicolumn{3}{|c|}{\textbf{Requisiti Facoltativi}} \\
    \hline

    F1 & Ottimizzazione delle prestazioni & Il modello dovrebbe essere in grado di gestire grandi volumi di codice senza degradare le prestazioni \\
    \hdashline

    F2 & Analisi di sicurezza & Il modello dovrebbe identificare potenziali vulnerabilità nel codice \\
    \hline

\end{tabularx}
\end{center}
\newpage