%----------------------------------------------------------------------------------------
%	STAGE DESCRIPTION
%----------------------------------------------------------------------------------------
\section{Scopo dello stage}
% Personalizzare inserendo lo scopo dello stage, cioè una breve descrizione
Il progetto di stage riguarda la creazione di un modello di intelligenza artificiale (LLM) in grado di supportare lo sviluppo e la qualità del codice prodotto dal team. Le fasi chiave del progetto includono:

- Analisi del codice sorgente prodotto dagli sviluppatori.

- Rilevamento delle non conformità rispetto a standard e best practice aziendali.

-  Segnalazione automatica degli errori individuati.

- Proposta ed eventuale esecuzione di correzioni automatiche.

Il risultato atteso è un modello LLM in grado di:

- Analizzare in modo accurato il codice prodotto dai nostri sviluppatori, identificando punti critici o potenziali errori.

- Individuare deviazioni dagli standard e dalle best practice, migliorando la qualità complessiva del software.

- Fornire segnalazioni chiare e comprensibili sugli errori rilevati, facilitando l’intervento dei programmatori.

- Proporre e, quando possibile, applicare correzioni automatiche per ridurre i tempi di debugging e aumentare l’efficienza del processo di sviluppo.

